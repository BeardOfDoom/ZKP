\chapter{Bevezetés}

Az entitások/felhasználók autorizácijó az internet terjedésével egyre gyakoribb folyamattá válik. A legelterjedtebb és legkézenfekvőbb módja ennek a jelszavak használata. A felhasználó a regisztráció során megad egy karaktersorozatot, amelyet a szerver eltárol, így belépéskor csupán össze kell hasonlítanunk, hogy a kapott jelszó megegyezik-e a tárolttal. Nem igényel bonyolult matematikai folyamatokat, letisztult folyamat. Azonban ez a módszer számos hátulütővel rendelkezik. Kezdjük azzal, hogy a http protokoll nem kínál titkosított kommunikációs csatornát, így egy közbeékelődéses (man-in-the-middle) támadással könnyedén kompromitálódhat jelszavunk és jelentős károkat szenvedhetünk. A https megjelenésével azonban a TLS/SSL rétegnek köszönhetően biztonságos, titkosított csatornákhoz juthatunk ezzel hátráltatva/megakadályozva a titkos adataink kompromitációját.

Ez azonban nem nyújt számunkra teljes védelmet, hiszem a támadások célpontjai közt szerepelnek a szerverek, amiken a jelszavaink tárolásra kerültek. Mivel egyre több szolgáltatás jelenik meg, újabb és újabb felhasználói fiókokat regisztrálunk és az esetek többségében a jelszavunk megegyezik. Ha bármelyik szolgáltatás ellen sikeres támadást halytanak végre, akkor ennek akár több felhasználói fiókunk is kárát szenvedheti.

Ezen probléma megoldására több előrelépés is született. Egyik lehetőség a kihívás-válasz (challenge-response) protokollok alkalmazása, amely esetén a tényleges titok nem kerül megosztásra, hanem a kihívó fél (szerver) kérdéseket tesz fel a kliens felé, aki válaszaival bizonyítani akarja kilétét. A kérdések utalnak a titokra, de nem konkrétan a titokra kérdeznek rá. Ez a kérdés-felelet folyamat addig zajlik, amíg a kihívó meg nem bizonyosodik arról, hogy a kliens valóban ismeri a titkos adatot. Egy egyszerű példa erre a jelszó alapú kihjvás-válasz protokollok egy fajtája, amely esetén a szerveren a jelszó hashelt változata kerül tárolásra, így a felhasználótól nem a jelszót kéri el, hanem annak a hash-ét és azt hasonlítja össze a tárolt adattal.

Sajnos ezeknek a protokolloknak is megvan a hibája. A működésük olyan, hogy minden egyes alkalmazásakor a titokról kiderül valamilyen információ, így alkalmazható sok esetben a választott-szöveg alapú támadás (chosen-text attack), így megfelelő kérdés-válasz párok alapján a támadó képes felhasználók megszemélyesítésére.

Egy másik lehetőség a korábbi problémára az egyre elterjedtebben alkalmazott többfaktoros azonosítási forma, amely esetén a jelszavak mellett valamilyen plusz adatot is kér a felhasználótól a szerver, például belépéskor SMS-ben küld egy rövid egyszer-használható jelszót. Ezzel megakadályozva a felhasználói fiókok kompromitációját abban az esetben is, ha a jelszóval már megtörtént. Többek között ezen autentikációs folyamatokra kínál az iparban népszerű szolgáltatást az Auth0\footnote{https://auth0.com/}.

A dolgozat azonban nem az előbbi opciók bemutatására koncentrál, hanem egy olyan módszerre, amely többet is kínál az előbbiektől, mégpedig abban, hogy képes a felhasználók teljes anonimitását is garantálni mindamellett, hogy az autentikációs folyamat során semmilyen információ nem kerül ki a felhasználói titokból. Ez a módszer a nulla-ismeretű protokollok alkalmazása.