\chapter{Nulla-ismeretű protokollok további alkalmazása}

Ezen fejezet célja, hogy az azonosításon túl további felhasználási lehetőségeit mutassam be a nulla-ismeretű protokolloknak. Tényleges protokollokat nem fogunk tekinteni, csupán ötleteket és azok előnyeit.

Mielőtt a következő protokollokat tárgyalnánk, szükséges megismerkednünk a \textit{tudásbizonyíték} mellett a \textit{tudásérv} (argument of knowledge) definícióval, ami annyiban különbözük, hogy a \textit{helyesség} tulajdonság esetén a statisztikai követelményt számítási követelménnyé gyengíti. Tehát, míg előbbi esetben a bizonyító fél korlátlan számítási kapacitással bír, addig utóbbiban polinomiálisan kötött számítási kapacitással számolunk.

\section{Nem-interaktív nulla-ismeretű protokollok}

A nulla-ismeretű protokollok közül eddig csupán interaktív protokollokról volt szó, azonban rendkívül elterjedt az interaktivitást nem követelő protokollok alkalmazása is. Ezen protokollok megszületését az a probléma indokolta, hogy az interaktivitás számos oda-vissza történő kommunikációt igényel és míg a számítások gyorsan elvégezhetők, az esetenként akár több száz adatcsere közel sem elhanyagolható kölcségekkel jár. 

Az ilyen protokollok \cite{NIZK, NIZK2} alapötlete, hogy létezzen valamilyen mindenkinek elérhető formában olyan random adat, amely rendelkezik a "well mixedness" tulajdonsággal, ami nem egy kézzel fogható tulajdonság, definíciót a szerzők se adtak. Felfoghatjuk úgy, hogy kriptográfiailag biztonságos random generátorral készült adat.

Ezzel a módszerrel a kommunikáció egyirányúvá válik, hiszen a hitelesítő fél feladata, hogy random értéket küldjön a bizonyító félnek helyettesítésre került a publikus random adattal, így minimális módosítással csökkenthettük protokollunk komplexitását, méghozzá úgy, hogy a nulla-ismeretű protokollok alaptulajdonságai nem sérülnek.

\section{Succinct non-interactive arguments of knowledge - SNARK}

Az interaktivitás elhagyásával sikerült kisebb komplexitással rendelkező protokollokat készíteni és ezt felhasználva a gyakorlatban is alkalmazni őket. Továbbra is probléma volt azonban, hogy sok esetben a bizonyító fél által küldött üzenet mérete nagy volt. Például, ha kis számítási kapacitással rendelkezünk és a számítást kiszervezzük valamilyen felhőszolgáltatónak, azonban szeretnénk valamilyen bizonyítékot is kapni a számítás helyességéről, akkor olyan protokoll alkalmazása szükséges használni, amely a kis számítási kapacitással bíró felhasználót nem terheli meg. Erre kompakt és számításigényét tekintve könnyű argumentumok alkalmazása szükséges. Ilyen protokollok például a SNARG (succinct non-interactive argument), vagy a SNARK (succinct non-interactive argument
of knowledge).

Egy kurrens protokoll a \cite{SNARK}, amelyet blockchain alapú kriptovaluták esetén előszeretettel alkalmaznak (Zcash\footnote{https://z.cash/technology/zksnarks/}, Filecoin\footnote{https://filecoin.io/}). Számos nyílt forráskódú megvalósítása létezik: \href{https://github.com/scipr-lab/libsnark}{libsnark}, \href{https://github.com/zkcrypto/bellman}{bellman}, \href{https://github.com/scipr-lab/dizk}{dizk}, \href{https://github.com/iden3/wasmsnark}{wasmsnark}.

\section{Bulletproof}

A bulletproof egy olyan nulla-imseretű protokoll, aminek megszületését főként a kriptovaluták elterjedése motiválta. A célja az, hogy megbízható inicializáló folyamat nélkül működjön a protokoll, gyorsan és kis méretű üzeneteket generálva. Ezeket a célokat teljesen kielégíte a \textit{BBBPWM17} \cite{BULLETPROOF}, mely alkalmazásainak fókuszában az áll, hogy a blockchain alapú tranzakciókhoz készítsen tartomány-igazoló bizonyítékot (range proof). Ezen protokollnak is több nyílt forrású implementációja létezik: \href{https://github.com/dalek-cryptography/bulletproofs}{dalek-bulletproofs}, \href{https://github.com/adjoint-io/bulletproofs}{adjoin.io-bulletproofs}.

Elena Nadolinski előadásában\footnote{https://docs.google.com/presentation/d/1gfB6WZMvM9mmDKofFibIgsyYShdf0RV\_Y8TLz3k1Ls0} szerepel egy összehasonlítása népszerű nulla-ismeretű protokolloknak, amely a \ref{Table::ZKPComparison} táblázatban látható.

\begin{table}[H]
\centering
    \begin{tabular}{|c|c|c|c|}
        \hline
         \thead{Protokoll} & \thead{Bizonyíték mérete} & \thead{Futásidő \\ (bizonyító)} & \thead{Futásidő \\ (hitelesítő)} \\ \hline
         \makecell{\textbf{SNARKs} \\ (megbízható inicializálást igényel)} & \textcolor{green}{288 byte} & \textcolor{orange}{2.3s} & \textcolor{green}{10ms} \\ \hline
        \textbf{STARKs} & \textcolor{red}{45KB-200KB} & \textcolor{green}{1.6s} & \textcolor{orange}{16ms} \\ \hline
        \textbf{Bulletproofs} & \textcolor{orange}{~1.3KB} & \textcolor{red}{30s} & \textcolor{red}{1100ms} \\ \hline
    \end{tabular}
    \caption{Modern nulla-ismeretű protokollok összehasonlítása.}
    \label{Table::ZKPComparison}
\end{table}