\chapter{Tobábbi nulla-ismeretű protokollok}

Ezen fejezet célja, hogy az azonosításon túl további felhasználási lehetőségeit mutassam be a nulla-ismeretű protokolloknak. Tényleges protokollokat nem fogunk tekinteni, csupán ötleteket és azok előnyeit.

\section*{Nem-interaktív nulla-ismeretű protokollok}

A nulla-ismeretű protokollok közül eddig csupán interaktív protokollokról volt szó, azonban rendkívül elterjedt az interaktivitást nem követelő protokollok alkalmazása is. Ezen protokollok megszületését az a probléma indokolta, hogy az interaktivitás számos oda-vissza történő kommunikációt igényel és míg a számítások gyorsan elvégezhetők, az esetenként akár több száz adatcsere közel sem elhanyagolható kölcségekkel jár. 

Az ilyen protokollok \cite{NIZK, NIZK2} alapötlete, hogy létezzen valamilyen mindenkinek elérhető formában olyan random adat, amely rendelkezik a "well mixedness" tulajdonsággal, ami nem egy kézzel fogható tulajdonság, definíciót a szerzők se adtak. Felfoghatjuk úgy, hogy kriptográfiailag biztonságos random generátorral készült adat.

Ezzel a módszerrel a kommunikáció egyirányúvá válik, hiszen a hitelesítő fél feladata, hogy random értéket küldjön a bizonyító félnek helyettesítésre került a publikus random adattal, így minimális módosítással csökkenthettük protokollunk komplexitását, méghozzá úgy, hogy a nulla-ismeretű protokollok alaptulajdonságai nem sérülnek.

\section*{???}

Mielőtt a következő protokollokat tárgyalnánk, szükséges megismerkednünk a \textit{proof of knowledge} mellett az \textit{argument of knowledge} definícióval, ami annyiban különbözük, hogy a \textit{megalapozottság} tulajdonság esetén a statisztikai követelményt számítási követelménnyé gyengíti. Tehát, míg előbbi esetben a bizonyító fél korlátlan számítási kapacitással bír, addig utóbbiban polinomiálisan kötött számítási kapacitással számolunk.

SNARKs \cite{SNARK}

STARKs \cite{STARK}

Bulletproofs \cite{BULLETPROOF}