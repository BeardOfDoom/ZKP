\chapter{Tobábbi nulla-ismeretű protokollok}

Ezen fejezet célja, hogy az azonosításon túl további felhasználási lehetőségeit mutassam be a nulla-ismeretű protokolloknak. Tényleges protokollokat nem fogunk tekinteni, csupán ötleteket és azok előnyeit.

NIZK \cite{NIZK}

Mielőtt a következő protokollokat tárgyalnánk, szükséges megismerkednünk a \textit{proof of knowledge} mellett az \textit{argument of knowledge} definícióval, ami annyiban különbözük, hogy a \textit{megalapozottság} tulajdonság esetén a statisztikai követelményt számítási követelménnyé gyengíti. Tehát, míg előbbi esetben a bizonyító fél korlátlan számítási kapacitással bír, addig utóbbiban polinomiálisan kötött számítási kapacitással számolunk.

SNARKs \cite{SNARK}

STARKs \cite{STARK}

Bulletproofs \cite{BULLETPROOF}